\documentclass[12pt,a4paper]{article}

\usepackage[a4paper,margin=1in]{geometry}
\usepackage{graphicx}
\usepackage{booktabs}
\usepackage{xcolor}
\usepackage{amsmath,amssymb}
\usepackage{enumitem}
\usepackage{fancyhdr}
\usepackage{titlesec}
\usepackage{array}

\definecolor{myblue}{RGB}{0,70,180}

\titleformat{\section}
  {\color{myblue}\Large\bfseries}
  {}
  {0pt}
  {}

\titleformat{\subsection}
  {\color{myblue}\large\bfseries}
  {}
  {0pt}
  {}

\pagestyle{fancy}
\fancyhf{}
\rfoot{}

\newcommand{\StudentName}{Keerthi.M}
\newcommand{\RollNo}{COMETFWC045}
\newcommand{\ExpDate}{\today}

\begin{document}

\noindent
\begin{minipage}[t]{0.55\textwidth}
  \vspace{0pt}
  \includegraphics[width=0.95\linewidth]{logo.jpeg}
\end{minipage}
\begin{minipage}[t]{0.44\textwidth}
  \vspace{0pt}
  \begin{flushright}
    \textbf{Name:} \StudentName\\
    \textbf{Roll No:} \RollNo\\
    \textbf{Date:} \ExpDate
  \end{flushright}
\end{minipage}

\vspace{6pt}

\begin{center}
{\color{myblue}\LARGE \textbf{NOR Gate Circuit --- Boolean Expression \& Hardware Verification}}
\end{center}

\vspace{8pt}


\section{Question}
\textbf \\
What is the Boolean expression for the output $f$ of the combinational logic circuit of NOR gates given below?

\begin{center}
\includegraphics[width=0.70\textwidth]{img2.jpeg}
\end{center}

\noindent \textbf{Required:} Find the Boolean expression for $f$ and verify it using hardware.

\section{Question Analysis}

\noindent All gates are NOR. For a NOR gate:
\[
\text{NOR}(X,Y) = \overline{X+Y}
\]

\subsection*{Top branch}
\[
X_1=\overline{P+Q},\quad X_2=\overline{Q+R}
\]
These feed a NOR:
\[
A=\overline{X_1+X_2}
=\overline{\overline{P+Q}+\overline{Q+R}}
=(P+Q)(Q+R)
\]

\subsection*{Bottom branch}
\[
Y_1=\overline{P+R},\quad Y_2=\overline{Q+R}
\]
These feed a NOR:
\[
B=\overline{Y_1+Y_2}
=\overline{\overline{P+R}+\overline{Q+R}}
=(P+R)(Q+R)
\]

\subsection*{Final NOR}
\[
f=\overline{A+B}
\]
\[
f=\overline{(P+Q)(Q+R)+(P+R)(Q+R)}
\]
Factor $(Q+R)$:
\[
f=\overline{(Q+R)\big((P+Q)+(P+R)\big)}
\]
\[
(P+Q)+(P+R)=P+Q+R
\]
Using absorption:
\[
(Q+R)(P+Q+R)=(Q+R)
\]
\[
\boxed{f=\overline{Q+R}}
\]

\section{Truth Table}

\begin{center}
\renewcommand{\arraystretch}{1.2}
\begin{tabular}{cccc}
\toprule
$P$ & $Q$ & $R$ & $f=\overline{Q+R}$ \\
\midrule
0 & 0 & 0 & 1 \\
0 & 0 & 1 & 0 \\
0 & 1 & 0 & 0 \\
0 & 1 & 1 & 0 \\
1 & 0 & 0 & 1 \\
1 & 0 & 1 & 0 \\
1 & 1 & 0 & 0 \\
1 & 1 & 1 & 0 \\
\bottomrule
\end{tabular}
\end{center}

\section{Hardware Implementation}
The above problem is implemented and tested in hardware using Arduino UNO board.
We used 7447 and a common anode seven segment display to display output $f$ as 0 or 1 for inputs $P,Q,R$ as per the truth table and verified the expression.

\section{Required Components \& Pin Connections}

\subsection{Components}
\begin{center}
\renewcommand{\arraystretch}{1.2}
\begin{tabular}{cl}
\toprule
S.No & Component \\
\midrule
1 & Arduino UNO Board \\
2 & Breadboard \\
3 & 7447 IC (BCD to 7-segment driver) \\
4 & Seven Segment Display (Common Anode) \\
5 & Resistors: 220$\Omega$ (for segments) \\
6 & Jumper Wires \\
7 & USB Cable \\
\bottomrule
\end{tabular}
\end{center}

\subsection{Arduino $\rightarrow$ 7447 (BCD Inputs)}
\begin{center}
\renewcommand{\arraystretch}{1.2}
\begin{tabular}{lll}
\toprule
7447 Input & 7447 Pin & Arduino Pin \\
\midrule
A (LSB) & 7 & D5 \\
B       & 1 & D6 \\
C       & 2 & D7 \\
D (MSB) & 6 & D8 \\
\bottomrule
\end{tabular}
\end{center}

\subsection{7447 Power \& Control Pins}
\begin{itemize}[leftmargin=1.2em]
\item 7447 pin 16 (VCC) $\rightarrow$ +5V
\item 7447 pin 8 (GND) $\rightarrow$ GND
\item pin 3 (LT) $\rightarrow$ +5V
\item pin 4 (BI/RBO) $\rightarrow$ +5V
\item pin 5 (RBI) $\rightarrow$ +5V
\end{itemize}

\subsection{7447 $\rightarrow$ Seven Segment (Common Anode)}
\begin{itemize}[leftmargin=1.2em]
\item Connect both COM pins of the 7-segment display $\rightarrow$ +5V
\item Connect 7447 outputs to segments \textbf{through 220$\Omega$ resistors}:
\end{itemize}

\begin{center}
\renewcommand{\arraystretch}{1.2}
\begin{tabular}{lll}
\toprule
7447 Output & 7447 Pin & Segment \\
\midrule
a & 13 & A \\
b & 12 & B \\
c & 11 & C \\
d & 10 & D \\
e & 9  & E \\
f & 15 & F \\
g & 14 & G \\
\bottomrule
\end{tabular}
\end{center}


\section{Logic Description}
\begin{itemize}[leftmargin=1.2em]
\item From Boolean simplification:
\[
f = \overline{Q+R}
\]
\item To show $f$ on 7-segment using 7447, we send BCD:
\[
f=0 \Rightarrow DCBA=0000 \Rightarrow \text{Display }0
\]
\[
f=1 \Rightarrow DCBA=0001 \Rightarrow \text{Display }1
\]
\end{itemize}

\section{Arduino Source Code (Auto Cycling)}

\begin{verbatim}
#include <Arduino.h>


const int A_pin = 5;  
const int B_pin = 6;   
const int C_pin = 7;   
const int D_pin = 8;   

void setup() {
  pinMode(A_pin, OUTPUT);
  pinMode(B_pin, OUTPUT);
  pinMode(C_pin, OUTPUT);
  pinMode(D_pin, OUTPUT);
}

void loop() {
  for (int n = 0; n < 8; n++) {

    int P = (n >> 2) & 1;
    int Q = (n >> 1) & 1;
    int R = (n >> 0) & 1;

  
    int f = !(Q || R);

   
    digitalWrite(A_pin, f);
    digitalWrite(B_pin, LOW);
    digitalWrite(C_pin, LOW);
    digitalWrite(D_pin, LOW);

    delay(1000);
  }
}


\section*{Setup / Output Image}
\begin{center}
\includegraphics[width=0.8\textwidth]{img1.jpeg}
\end{center}
\end{verbatim}


\section{Experimental Truth Table}

\begin{center}
\renewcommand{\arraystretch}{1.2}
\begin{tabular}{cccc}
\toprule
$P$ & $Q$ & $R$ & Observed $f$ (7-seg) \\
\midrule
0 & 0 & 0 & 1 \\
0 & 0 & 1 & 0 \\
0 & 1 & 0 & 0 \\
0 & 1 & 1 & 0 \\
1 & 0 & 0 & 1 \\
1 & 0 & 1 & 0 \\
1 & 1 & 0 & 0 \\
1 & 1 & 1 & 0 \\
\bottomrule
\end{tabular}
\end{center}


\section{Conclusion}
From the truth table and hardware verification, the NOR circuit output is confirmed as:
\[
\boxed{f=\overline{Q+R}}
\]
Hence the correct option is $\overline{Q+R}$.

\end{document}
