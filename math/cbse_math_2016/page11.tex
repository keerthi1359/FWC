% ---------------- PAGE 11 ----------------
\setcounter{page}{11}

\begin{enumerate}[label=\textbf{\arabic*.}, start=25, leftmargin=*, itemsep=22pt]

\item {\hindifont सिद्ध कीजिए कि वक्र $y^2=4x$ और $x^2=4y$, उस वर्ग के क्षेत्रफल को तीन बराबर भागों में बाँटते हैं जो कि रेखाओं $x=0,\;x=4,\;y=4$ और $y=0$ द्वारा परिबद्ध है।}\\[10pt]
Prove that the curves $y^2=4x$ and $x^2=4y$ divide the area of square bounded by $x=0,\;x=4,\;y=4$ and $y=0$ into three equal parts.

\item {\hindifont सारिणिकों के गुणधर्मों का प्रयोग कर सिद्ध कीजिए कि $\triangle ABC$ एक समद्विबाहु त्रिभुज है यदि}\\[6pt]
\[
\begin{vmatrix}
1 & 1 & 1\\
1+\cos A & 1+\cos B & 1+\cos C\\
\cos^2 A+\cos A & \cos^2 B+\cos B & \cos^2 C+\cos C
\end{vmatrix}=0
\]
\begin{center}
{\hindifont अथवा}\\[4pt]
\end{center}

{\hindifont एक दुकानदार के पास तीन विभिन्न प्रकार के पेन ‘A’, ‘B’ और ‘C’ हैं। मीनू ने प्रत्येक प्रकार का एक-एक पेन कुल ₹ 21 में खरीदा। जीवन ने ‘A’ प्रकार के 4 पेन, ‘B’ प्रकार के 3 पेन और ‘C’ प्रकार के 2 पेन ₹ 60 में खरीदे जबकि शिखा ने ‘A’ प्रकार के 6 पेन, ‘B’ प्रकार के 2 पेन और ‘C’ प्रकार के 3 पेन ₹ 70 में खरीदे। आव्यूह विधि से प्रत्येक प्रकार के पेन का मूल्य ज्ञात कीजिए।}\\[10pt]



Using properties of determinants, show that $\triangle ABC$ is isosceles if :
\[
\begin{vmatrix}
1 & 1 & 1\\
1+\cos A & 1+\cos B & 1+\cos C\\
\cos^2 A+\cos A & \cos^2 B+\cos B & \cos^2 C+\cos C
\end{vmatrix}=0
\]

\begin{center}
\textbf{OR}
\end{center}

A shopkeeper has 3 varieties of pens ‘A’, ‘B’ and ‘C’. Meenu purchased 1 pen of each variety for a total of Rs-21. Jeevan purchased 4 pens of ‘A’ variety, 3 pens of ‘B’ variety and 2 pens of ‘C’ variety for Rs-60. While Shikha purchased 6 pens of ‘A’ variety, 2 pens of ‘B’ variety and 3 pens of ‘C’ variety for ₹ 70. Using matrix method, find cost of each variety of pen.

\end{enumerate}

\vfill
\noindent
65/1/C\hfill 
