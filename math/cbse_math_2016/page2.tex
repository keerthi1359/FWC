% ---------------- PAGE 2 ----------------
% (page number will be handled by paper.tex counter unless you set it manually)

\vspace*{8mm}

% -------- Hindi Instructions ----------
{\textHindi{\textbf{सामान्य निर्देश :}}}

\vspace{3mm}

\begin{enumerate}[label=(\roman*), leftmargin=*, itemsep=6pt]
  \item \textHindi{सभी प्रश्न अनिवार्य हैं।}
  \item \textHindi{कृपया जाँच कर लें कि इस प्रश्न-पत्र में 26 प्रश्न हैं।}
  \item \textHindi{खंड अ के प्रश्न 1 - 6 तक अति लघु-उत्तर वाले प्रश्न हैं और प्रत्येक प्रश्न के लिए 1 अंक निर्धारित है।}
  \item \textHindi{खंड ब के प्रश्न 7 - 19 तक दीर्घ-उत्तर I प्रकार के प्रश्न हैं और प्रत्येक प्रश्न के लिए 4 अंक निर्धारित है।}
  \item \textHindi{खंड स के प्रश्न 20 - 26 तक दीर्घ-उत्तर II प्रकार के प्रश्न हैं और प्रत्येक प्रश्न के लिए 6 अंक निर्धारित है।}
  \item \textHindi{उत्तर लिखना प्रारम्भ करने से पहले कृपया प्रश्न का क्रमांक अवश्य लिखिए।}
\end{enumerate}

\vspace{10mm}

% -------- English Instructions ----------
{\textbf{General Instructions :}}

\vspace{10mm}

\begin{enumerate}[label=(\roman*), leftmargin=*, itemsep=6pt]
  \item All questions are compulsory.
   
  \item Please check that this question paper contains 26 questions.
   
  \item Questions 1 - 6 in Section A are very short-answer type questions carrying 1 mark each.
   
  \item Questions 7 - 19 in Section B are long-answer I type questions carrying 4 marks each.
   
  \item Questions 20 - 26 in Section C are long-answer II type questions carrying 6 marks each.
  
  \item Please write down the serial number of the question before attempting it.
\end{enumerate}

\vfill

% Footer (same style as your paper)
\noindent
65/1/C\hfill \hfill
