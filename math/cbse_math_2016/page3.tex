% ---------------- PAGE 3 ----------------
\setcounter{page}{3}

% Headings
\begin{center}
  {\textHindi{\textbf{खंड - अ}}}\\[2mm]
  {\Large\textbf{SECTION - A}}\\[10mm]
  {\textHindi{\textbf{प्रश्न संख्या 1 से 6 तक प्रत्येक प्रश्न का 1 अंक है।}}}\\[2mm]
  {\textbf{Question numbers 1 to 6 carry 1 mark each.}}
\end{center}

\vspace{10mm}

% Questions
\begin{enumerate}[leftmargin=*, label=\arabic*., itemsep=14mm]

  \item
  \textHindi{यदि $x \in \mathbb{N}$ और
  $\left|\begin{matrix}x+3 & -2\\ -3x & 2x\end{matrix}\right| = 8$ है, तो $x$ का मान ज्ञात कीजिए।}\\[3mm]
  If $x \in \mathbb{N}$ and
  $\left|\begin{matrix}x+3 & -2\\ -3x & 2x\end{matrix}\right| = 8$,
  then find the value of $x$.

  \item
  \textHindi{प्रारम्भिक स्तम्भ संक्रिया $C_{2} \rightarrow C_{2}+2C_{1}$ निम्न आव्यूह समीकरण पर लगाइए :}\\[6mm]
  \[
  \begin{pmatrix} 2 & 1 \\ 2 & 0 \end{pmatrix}
  =
  \begin{pmatrix} 3 & 1 \\ 2 & 0 \end{pmatrix}
  \begin{pmatrix} 1 & 0 \\ -1 & 1 \end{pmatrix}
  \]
  \vspace{6mm}
  Use elementary column operation $C_{2} \rightarrow C_{2}+2C_{1}$ in the following matrix equation :\\[6mm]
  \[
  \begin{pmatrix} 2 & 1 \\ 2 & 0 \end{pmatrix}
  =
  \begin{pmatrix} 3 & 1 \\ 2 & 0 \end{pmatrix}
  \begin{pmatrix} 1 & 0 \\ -1 & 1 \end{pmatrix}
  \]

  \item
  \textHindi{कोटि $2 \times 2$ के सभी संभव आव्यूहों की संख्या, जिनका प्रत्येक अवयव 1, 2 अथवा 3 है, लिखिए।}\\[3mm]
  Write the number of all possible matrices of order $2\times 2$
  with each entry 1, 2 or 3.

\end{enumerate}

\vfill

% Footer
\noindent
65/1/C\hfill \hfill \textbf{P.T.O.}
