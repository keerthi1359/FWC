% ---------------- PAGE 10 ----------------
\setcounter{page}{10}

\begin{enumerate}[label=\textbf{\arabic*.}, start=22, leftmargin=*, itemsep=22pt]

\item {\hindifont बिंदु $P$, जिसका स्थिति सदिश $2\hat{i}+3\hat{j}+4\hat{k}$ है से समतल
$\vec{r}\cdot(2\hat{i}+\hat{j}+3\hat{k})-26=0$ पर खींचे गए लम्ब के पाद का स्थिति सदिश तथा लम्बवत दूरी ज्ञात कीजिए। तल में $P$ का प्रतिबिम्ब भी ज्ञात कीजिए।}\\[10pt]
Find the position vector of the foot of perpendicular and the perpendicular distance from the point $P$ with position vector $2\hat{i}+3\hat{j}+4\hat{k}$ to the plane
$\vec{r}\cdot(2\hat{i}+\hat{j}+3\hat{k})-26=0$. Also find image of $P$ in the plane.

\item {\hindifont दर्शाइए कि एक द्विआधारी संक्रिया जो $A=\mathbb{R}-\{-1\}$ पर सभी $a,b\in A$ के लिए
$a*b=a+b+ab$ द्वारा परिभाषित है, क्रम विनिमेय तथा साहचर्य है। $A$ में $*$ का तटस्थ अवयव ज्ञात कीजिए तथा सिद्ध कीजिए कि $A$ का प्रत्येक अवयव व्युत्क्रमणीय है।}\\[20pt]
Show that the binary operation $*$ on $A=\mathbb{R}-\{-1\}$ defined as $a*b=a+b+ab$ for all $a,b\in A$ is commutative and associative on $A$. Also find the identity element of $*$ in $A$ and prove that every element of $A$ is invertible.

\item {\hindifont सिद्ध कीजिए कि समद्विबाहु त्रिभुज, जिसमें $r$ त्रिज्या का एक अंतर्वृत खींचा गया है, का न्यूनतम परिमाप $6\sqrt{3}\,r$ है।}\\[10pt]
\begin{center}
{\hindifont अथवा}\\[4pt]
\end{center}
{\hindifont यदि एक समकोण त्रिभुज में कर्ण तथा एक भुजा का योग दिया गया हो, तो दर्शाइए कि त्रिभुज का क्षेत्रफल अधिकतम होगा जबकि उनके बीच का कोण $\dfrac{\pi}{3}$ होगा।}\\[10pt]




Prove that the least perimeter of an isosceles triangle in which a circle of radius $r$ can be inscribed is $6\sqrt{3}\,r$.
\begin{center}
\textbf{OR}
\end{center}
If the sum of lengths of hypotenuse and a side of a right angled triangle is given, show that area of triangle is maximum, when the angle between them is $\dfrac{\pi}{3}$.

\end{enumerate}

\vfill
\noindent
65/1/C\hfill 
