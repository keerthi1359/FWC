% ---------------- PAGE 4 ----------------
\setcounter{page}{4}

\vspace*{6mm}

\begin{enumerate}[leftmargin=*, label=\arabic*., start=4, itemsep=14mm]

  \item
  \textHindi{उस बिंदु का स्थिति सदिश लिखिए जो बिंदुओं, जिनके स्थिति सदिश
  $\vec{3a}-\vec{2b}$ तथा $\vec{2a}+\vec{3b}$ हैं, को मिलाने वाले रेखाखंड को
  $2:1$ के अनुपात में बाँटता है।}\\[3mm]
  Write the position vector of the point which divides the join of points
  with position vectors $\vec{3a}-\vec{2b}$ and $\vec{2a}+\vec{3b}$ in the ratio $2:1$.

  \item
  \textHindi{उन मात्रक सदिशों की संख्या लिखिए जो सदिशों
  $\vec{a}=2\hat{i}+\hat{j}+2\hat{k}$ तथा
  $\vec{b}=\hat{j}+\hat{k}$ दोनों पर लम्ब हैं।}\\[3mm]
  Write the number of vectors of unit length perpendicular to both the vectors
  $\vec{a}=2\hat{i}+\hat{j}+2\hat{k}$ and $\vec{b}=\hat{j}+\hat{k}$.

  \item
  \textHindi{उस समतल का सदिश समीकरण ज्ञात कीजिए, जो कि
  $x, y$ और $z$-अक्ष पर क्रमशः $3, -4$ और $2$ अंतःखंड काटता है।}\\[3mm]
  Find the vector equation of the plane with intercepts $3$, $-4$ and $2$
  on $x$, $y$ and $z$-axis respectively.

\end{enumerate}

\vspace{12mm}

% -------- SECTION B --------
\begin{center}
  {\textHindi{\textbf{खंड – ब}}}\\[2mm]
  {\Large\textbf{SECTION - B}}\\[10mm]
  {\textHindi{\textbf{प्रश्न संख्या 7 से 19 तक प्रत्येक प्रश्न के 4 अंक हैं।}}}\\[2mm]
  {\textbf{Question numbers 7 to 19 carry 4 marks each.}}
\end{center}

\vspace{12mm}

\begin{enumerate}[leftmargin=*, label=\arabic*., start=7, itemsep=10mm]

  \item
  \textHindi{उस बिंदु के निर्देशांक कीजिए जहाँ पर बिंदुओं
  $A(3,4,1)$ और $B(5,1,6)$ से होकर जाने वाली रेखा $XZ$ समतल को प्रतिच्छेद करती है।
  वह कोण भी ज्ञात कीजिए जो यह रेखा $XZ$ समतल के साथ बनाती है।}\\[3mm]
  Find the coordinates of the point where the line through the points
  $A(3,4,1)$ and $B(5,1,6)$ crosses the $XZ$ plane.
  Also find the angle which this line makes with the $XZ$ plane.

\end{enumerate}

\vfill

% Footer
\noindent
65/1/C\hfill \hfill

