% ---------------- PAGE 9 ----------------
\setcounter{page}{9}

\begin{center}
{\hindifont खण्ड - स}\\[3pt]
\textbf{SECTION - C}\\[10pt]
{\hindifont प्रश्न संख्या 20 से 26 तक प्रत्येक प्रश्न के 6 अंक हैं।}\\[4pt]
\textbf{Question numbers 20 to 26 carry 6 marks each.}
\end{center}

\vspace{18pt}

\begin{enumerate}[label=\textbf{\arabic*.}, start=20, leftmargin=*, itemsep=22pt]

\item {\hindifont दो प्रकार के खाद ‘A’ और ‘B’ हैं। ‘A’ में 12\% नाइट्रोजन और 5\% फास्फोरिक एसिड है जबकि ‘B’ में 4\% नाइट्रोजन और 5\% फास्फोरिक एसिड है। मिट्टी के स्थिति परीक्षण के बाद किसान को ज्ञात हुआ कि उसे फसल के लिए कम से कम 12 कि.ग्रा. नाइट्रोजन और 12 कि.ग्रा. फास्फोरिक एसिड की आवश्यकता है। यदि ‘A’ का मूल्य ₹ 10 प्रति कि.ग्रा. और ‘B’ का मूल्य ₹ 8 प्रति कि.ग्रा. है तो आलेख द्वारा परिकलन कीजिए कि उसे प्रत्येक प्रकार की कितनी खाद प्रयोग करनी चाहिए कि कम से कम कीमत में पोषक तत्वों की आवश्यकता पूरी हो जाए।}\\[10pt]
There are two types of fertilisers ‘A’ and ‘B’. ‘A’ consists of 12\% nitrogen and 5\% phosphoric acid whereas ‘B’ consists of 4\% nitrogen and 5\% phosphoric acid. After testing the soil conditions, farmer finds that he needs at least 12 kg of nitrogen and 12 kg of phosphoric acid for his crops. If ‘A’ costs ₹ 10 per kg and ‘B’ cost ₹ 8 per kg, then graphically determine how much of each type of fertiliser should be used so that nutrient requirements are met at a minimum cost.

\item {\hindifont 20 अच्छे संतरे में 5 खराब संतरे आकस्मिक कारण से मिल गए हैं। चार संतरे उत्तरोत्तर प्रतिस्थापन के सहित निकाले गए, तो खराब संतरे को निकालने की संख्या का प्रायिकता बंटन ज्ञात कीजिए। बंटन का माध्य तथा प्रसरण भी ज्ञात कीजिए।}\\[10pt]
Five bad oranges are accidently mixed with 20 good ones. If four oranges are drawn one by one successively with replacement, then find the probability distribution of number of bad oranges drawn. Hence find the mean and variance of the distribution.

\end{enumerate}

\vfill
\noindent
65/1/C\hfill \hfill \textbf{P.T.O.}
