\documentclass[12pt,a4paper]{article}

\usepackage[a4paper,margin=1in]{geometry}
\usepackage{graphicx}
\usepackage{booktabs}
\usepackage{xcolor}
\usepackage{amsmath,amssymb}
\usepackage{enumitem}
\usepackage{fancyhdr}
\usepackage{titlesec}
\usepackage{array}


\definecolor{myblue}{RGB}{0,70,180}

\titleformat{\section}
  {\color{myblue}\Large\bfseries}
  {}
  {0pt}
  {}

\titleformat{\subsection}
  {\color{myblue}\large\bfseries}
  {}
  {0pt}
  {}

\pagestyle{fancy}
\fancyhf{}
\rfoot{


}


\newcommand{\StudentName}{Keerthi.M}
\newcommand{\RollNo}{COMETFWC045}
\newcommand{\ExpDate}{\today}

\begin{document}


\noindent
\begin{minipage}[t]{0.55\textwidth}
  \vspace{0pt}
  \includegraphics[width=0.95\linewidth]{logo.jpeg}
\end{minipage}
\begin{minipage}[t]{0.44\textwidth}
  \vspace{0pt}
  \begin{flushright}
    \textbf{Name:} \StudentName\\
    \textbf{Roll No:} \RollNo\\
    \textbf{Date:} \ExpDate
  \end{flushright}
\end{minipage}

\vspace{6pt}


\begin{center}
{\color{myblue}\LARGE \textbf{GATE 2010 CS, $9^{th}$ Question Analysis}}
\end{center}

\vspace{8pt}

\section{Question 9}
\textbf{(GATE 2010 CS, Question 9)}\\
The Boolean expression for the output $f$ of the multiplexer shown below is:

\begin{center}
\includegraphics[width=0.75\textwidth]{q9.jpeg}
\end{center}

\noindent
\textbf{Given:}
\begin{itemize}[leftmargin=1.2em]
\item Select lines: $S_1=P$, $S_0=Q$
\item Data inputs: $I_0=R$, $I_1=\overline{R}$, $I_2=\overline{R}$, $I_3=R$
\end{itemize}

\noindent
\textbf{Required:} Find the Boolean expression for $f$ and verify it using hardware.


\section{Question Analysis}

\begin{itemize}[leftmargin=1.2em]
\item Standard 4:1 MUX output equation:
\[
f = I_0\overline{S_1}\overline{S_0}+I_1\overline{S_1}S_0+I_2S_1\overline{S_0}+I_3S_1S_0
\]
\item Substitute $S_1=P$, $S_0=Q$:
\[
f = I_0\overline{P}\overline{Q}+I_1\overline{P}Q+I_2P\overline{Q}+I_3PQ
\]
\item Substitute $I_0=R$, $I_1=\overline{R}$, $I_2=\overline{R}$, $I_3=R$:
\[
f = R\overline{P}\overline{Q}+\overline{R}\,\overline{P}Q+\overline{R}\,P\overline{Q}+RPQ
\]
\item Simplified result:
\[
\boxed{f = P \oplus Q \oplus R}
\]
\end{itemize}


\section{Truth Table}

\begin{center}
\renewcommand{\arraystretch}{1.2}
\begin{tabular}{cccc}
\toprule
$P$ & $Q$ & $R$ & $f=P\oplus Q\oplus R$ \\
\midrule
0 & 0 & 0 & 0 \\
0 & 0 & 1 & 1 \\
0 & 1 & 0 & 1 \\
0 & 1 & 1 & 0 \\
1 & 0 & 0 & 1 \\
1 & 0 & 1 & 0 \\
1 & 1 & 0 & 0 \\
1 & 1 & 1 & 1 \\
\bottomrule
\end{tabular}
\end{center}

\section{Hardware Implementation}
The above problem is implemented and tested in hardware using Arduino UNO board. Here we used 7447 and seven segment to display output F is 1 or 0 for input A B C as per truth table and verified the expression.

\section{Required Components \& Pin Connections}

\subsection{Components}
\begin{center}
\renewcommand{\arraystretch}{1.2}
\begin{tabular}{cl}
\toprule
S.No & Component \\
\midrule
1 & Arduino UNO Board \\
2 & Breadboard \\
3 & 7447 IC (BCD to 7-segment driver) \\
4 & Seven Segment Display (Common Anode) \\
5 & Resistors: 220$\Omega$ (for segments) \\
6 & Jumper Wires \\
7 & USB Cable \\
\bottomrule
\end{tabular}
\end{center}

\subsection{Arduino $\rightarrow$ 7447 (BCD Inputs)}
\begin{center}
\renewcommand{\arraystretch}{1.2}
\begin{tabular}{lll}
\toprule
7447 Input & 7447 Pin & Arduino Pin \\
\midrule
A (LSB) & 7 & D5 \\
B       & 1 & D6 \\
C       & 2 & D7 \\
D (MSB) & 6 & D8 \\
\bottomrule
\end{tabular}
\end{center}

\subsection{7447 Power \& Control Pins}
\begin{itemize}[leftmargin=1.2em]
\item 7447 pin 16 (VCC) $\rightarrow$ +5V
\item 7447 pin 8 (GND) $\rightarrow$ GND
\item pin 3 (LT) $\rightarrow$ +5V
\item pin 4 (BI/RBO) $\rightarrow$ +5V
\item pin 5 (RBI) $\rightarrow$ +5V
\end{itemize}

\subsection{7447 $\rightarrow$ Seven Segment (Common Anode)}
\begin{itemize}[leftmargin=1.2em]
\item Connect both COM pins of the 7-segment display $\rightarrow$ +5V
\item Connect 7447 outputs to segments \textbf{through 220$\Omega$ resistors}:
\end{itemize}

\begin{center}
\renewcommand{\arraystretch}{1.2}
\begin{tabular}{lll}
\toprule
7447 Output & 7447 Pin & Segment \\
\midrule
a & 13 & A \\
b & 12 & B \\
c & 11 & C \\
d & 10 & D \\
e & 9  & E \\
f & 15 & F \\
g & 14 & G \\
\bottomrule
\end{tabular}
\end{center}


\section{Logic Description}
\begin{itemize}[leftmargin=1.2em]
\item The MUX output is:
\[
f = P\oplus Q\oplus R
\]
\item To show $f$ on 7-segment using 7447, we send BCD:
\[
f=0 \Rightarrow DCBA=0000 \Rightarrow \text{Display }0
\]
\[
f=1 \Rightarrow DCBA=0001 \Rightarrow \text{Display }1
\]
\end{itemize}


\section{Arduino Source Code (Auto Cycling)}

\noindent\textbf{Note:} This program cycles all $PQR$ combinations from 000 to 111 and displays $f$.

\begin{verbatim}


const int A_pin = 5;   // -> 7447 A (pin 7)
const int B_pin = 6;   // -> 7447 B (pin 1)
const int C_pin = 7;   // -> 7447 C (pin 2)
const int D_pin = 8;   // -> 7447 D (pin 6)

void setup() {
  pinMode(A_pin, OUTPUT);
  pinMode(B_pin, OUTPUT);
  pinMode(C_pin, OUTPUT);
  pinMode(D_pin, OUTPUT);
}

void loop() {
  for (int n = 0; n < 8; n++) {
    int P = (n >> 2) & 1;
    int Q = (n >> 1) & 1;
    int R = (n >> 0) & 1;

    int f = P ^ Q ^ R;

    // f=0 -> BCD 0000, f=1 -> BCD 0001
    digitalWrite(A_pin, f);
    digitalWrite(B_pin, LOW);
    digitalWrite(C_pin, LOW);
    digitalWrite(D_pin, LOW);

    delay(1000);
  }
}
\end{verbatim}


\section*{Setup / Output Image}
\begin{center}
\includegraphics[width=0.8\textwidth]{img1.jpeg}
\end{center}

\section{Experimental Truth Table}
\begin{center}
\renewcommand{\arraystretch}{1.2}
\begin{tabular}{cccc}
\toprule
$P$ & $Q$ & $R$ & Observed $f$ (7-seg) \\
\midrule
0 & 0 & 0 & 0 \\
0 & 0 & 1 & 1 \\
0 & 1 & 0 & 1 \\
0 & 1 & 1 & 0 \\
1 & 0 & 0 & 1 \\
1 & 0 & 1 & 0 \\
1 & 1 & 0 & 0 \\
1 & 1 & 1 & 1 \\
\bottomrule
\end{tabular}
\end{center}

\section{Conclusion}
From the truth table and hardware verification, the multiplexer output is confirmed as:
\[
\boxed{f = P \oplus Q \oplus R}
\]
Hence the correct option is the XOR of all three inputs.

\end{document}
