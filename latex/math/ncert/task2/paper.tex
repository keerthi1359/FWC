\documentclass[11pt,a4paper]{article}

\usepackage[a4paper, top=2.5cm, bottom=2.5cm, left=2cm, right=2cm]{geometry}

\usepackage{fontspec}
\setmainfont{Times New Roman}

\usepackage{amsmath, amsfonts, amssymb}
\usepackage{enumitem}

\usepackage{booktabs}
\usepackage{fancyhdr}
\usepackage{graphicx}
\usepackage{float} 

\usepackage{xcolor}
\definecolor{cometblue}{RGB}{0, 75, 145}
\definecolor{cometgrey}{RGB}{128, 128, 128}

% Fix fancyhdr warning
\setlength{\headheight}{14pt}

\pagestyle{fancy}
\fancyhf{}
\fancyhead[C]{\textit{Mathematics}}
\fancyfoot[C]{\thepage}
\renewcommand{\headrulewidth}{0.4pt}

\begin{document}

\thispagestyle{empty}

% =================== TOP HEADER (NAME/ID/DATE + LOGO) ===================
\noindent
\begin{minipage}[t]{0.6\textwidth}
    \raggedright
    \begin{tabular}{@{}rl@{}}
        \textbf{Name:} & Keerthi \\
        \textbf{ID:}   & cometfwc045 \\
        \textbf{Date:} & 20/01/2026
    \end{tabular}
\end{minipage}
\begin{minipage}[t]{0.4\textwidth}
    \raggedleft
    \includegraphics[width=3.5cm]{logo.jpeg}
\end{minipage}

\vspace{0.8cm}

% =================== EXERCISE 6.1 TITLE ===================
\begin{center}
    {\color{cometblue}\hrule height 1.5pt}
    \vspace{0.3cm}
    \LARGE \textbf{EXERCISE 6.1} \\
    \large \textit{Application of Derivatives: Rates of Change}
    \vspace{0.3cm}
    {\color{cometblue}\hrule height 1pt}
\end{center}

\vspace{0.5cm}

\begin{enumerate}[leftmargin=*]
    \item Find the rate of change of the area of a circle with respect to its radius $r$ when
    \begin{itemize}
        \item[(a)] $r = 3\text{ cm}$ \hfill (b) $r = 4\text{ cm}$
    \end{itemize}

    \item The volume of a cube is increasing at the rate of $8\text{ cm}^3/\text{s}$. How fast is the surface area increasing when the length of an edge is $12\text{ cm}$?

    \item The radius of a circle is increasing uniformly at the rate of $3\text{ cm}/\text{s}$. Find the rate at which the area of the circle is increasing when the radius is $10\text{ cm}$.

    \item An edge of a variable cube is increasing at the rate of $3\text{ cm}/\text{s}$. How fast is the volume of the cube increasing when the edge is $10\text{ cm}$ long?

    \item A stone is dropped into a quiet lake and waves move in circles at the speed of $5\text{ cm}/\text{s}$. At the instant when the radius of the circular wave is $8\text{ cm}$, how fast is the enclosed area increasing?

    \item The radius of a circle is increasing at the rate of $0.7\text{ cm}/\text{s}$. What is the rate of increase of its circumference?

    \item The length $x$ of a rectangle is decreasing at the rate of $5\text{ cm}/\text{minute}$ and the width $y$ is increasing at the rate of $4\text{ cm}/\text{minute}$.
    When $x = 8\text{ cm}$ and $y = 6\text{ cm}$, find the rates of change of
    \begin{itemize}
        \item[(a)] the perimeter \hfill (b) the area of the rectangle
    \end{itemize}

    \item A balloon, which always remains spherical on inflation, is being inflated by pumping in $900$ cubic centimetres of gas per second.
    Find the rate at which the radius of the balloon increases when the radius is $15\text{ cm}$.

    \item A balloon, which always remains spherical, has a variable radius. Find the rate at which its volume is increasing with the radius when the latter is $10\text{ cm}$.

    \item A ladder $5\text{ m}$ long is leaning against a wall. The bottom of the ladder is pulled along the ground, away from the wall, at the rate of $2\text{ cm}/\text{s}$.
    How fast is its height on the wall decreasing when the foot of the ladder is $4\text{ m}$ away from the wall?

    \item A particle moves along the curve $6y = x^3 + 2$. Find the points on the curve at which the $y$-coordinate is changing $8$ times as fast as the $x$-coordinate.

    \item The radius of an air bubble is increasing at the rate of $\frac{1}{2}\text{ cm}/\text{s}$. At what rate is the volume of the bubble increasing when the radius is $1\text{ cm}$?

    \item A balloon, which always remains spherical, has a variable diameter $\frac{3}{2}(2x + 1)$. Find the rate of change of its volume with respect to $x$.

    \item Sand is pouring from a pipe at the rate of $12\text{ cm}^3/\text{s}$. The falling sand forms a cone on the ground in such a way that the height of the cone is always one-sixth of the radius of the base.
    How fast is the height of the sand cone increasing when the height is $4\text{ cm}$?

    \item The total cost $C(x)$ in Rupees associated with the production of $x$ units of an item is given by
    $$C(x) = 0.007x^3 - 0.003x^2 + 15x + 4000.$$
    Find the marginal cost when $17$ units are produced.

    \item The total revenue in Rupees received from the sale of $x$ units of a product is given by
    $$R(x) = 13x^2 + 26x + 15.$$
    Find the marginal revenue when $x = 7$.

    \item The rate of change of the area of a circle with respect to its radius $r$ at $r = 6\text{ cm}$ is:
    \begin{itemize}
        \item[(A)] $10\pi$ \hfill (B) $12\pi$ \hfill (C) $8\pi$ \hfill (D) $11\pi$
    \end{itemize}

    \item The total revenue in Rupees received from the sale of $x$ units of a product is given by
    $$R(x) = 3x^2 + 36x + 5.$$
    The marginal revenue, when $x = 15$ is:
    \begin{itemize}
        \item[(A)] $116$ \hfill (B) $96$ \hfill (C) $90$ \hfill (D) $126$
    \end{itemize}
\end{enumerate}

\newpage

\fancyhead[C]{\textit{Mathematics: Exercise 6.2}}

\begin{center}
    {\color{cometblue}\hrule height 1.5pt}
    \vspace{0.3cm}
    \LARGE \textbf{EXERCISE 6.2} \\
    \large \textit{Increasing and Decreasing Functions}
    \vspace{0.3cm}
    {\color{cometblue}\hrule height 1pt}
\end{center}

\vspace{0.5cm}

\begin{enumerate}[leftmargin=*]
    \item Show that the function given by $f(x) = 3x + 17$ is increasing on $\mathbb{R}$.
    \item Show that the function given by $f(x) = e^{2x}$ is increasing on $\mathbb{R}$.
    \item Show that the function given by $f(x) = \sin x$ is
    \begin{itemize}
        \item[(a)] increasing in $\left(0, \frac{\pi}{2}\right)$ \hfill (b) decreasing in $\left(\frac{\pi}{2}, \pi\right)$
        \item[(c)] neither increasing nor decreasing in $(0, \pi)$
    \end{itemize}

    \item Find the intervals in which the function $f$ given by $f(x) = 2x^2 - 3x$ is
    \begin{itemize}
        \item[(a)] increasing \hfill (b) decreasing
    \end{itemize}

    \item Find the intervals in which the function $f$ given by $f(x) = 2x^3 - 3x^2 - 36x + 7$ is
    \begin{itemize}
        \item[(a)] increasing \hfill (b) decreasing
    \end{itemize}

    \item Find the intervals in which the following functions are strictly increasing or decreasing:
    \begin{itemize}
        \item[(a)] $x^2 + 2x - 5$ \hfill (b) $10 - 6x - 2x^2$
        \item[(c)] $-2x^3 - 9x^2 - 12x + 1$ \hfill (d) $6 - 9x - x^2$
        \item[(e)] $(x + 1)^3(x - 3)^3$
    \end{itemize}

    \item Show that $y = \log(1+x) - \frac{2x}{2+x}$, $x > -1$, is an increasing function of $x$ throughout its domain.

    \item Find the values of $x$ for which $y = [x(x-2)]^2$ is an increasing function.

    \item Prove that $y = \frac{4\sin\theta}{2+\cos\theta} - \theta$ is an increasing function of $\theta$ in $\left[0, \frac{\pi}{2}\right]$.

    \item Prove that the logarithmic function is increasing on $(0, \infty)$.

    \item Prove that the function $f$ given by $f(x) = x^2 - x + 1$ is neither strictly increasing nor decreasing on $(-1, 1)$.

    \item Which of the following functions are decreasing on $\left(0, \frac{\pi}{2}\right)$?
    \begin{itemize}
        \item[(A)] $\cos x$ \hfill (B) $\cos 2x$ \hfill (C) $\cos 3x$ \hfill (D) $\tan x$
    \end{itemize}

    \item On which of the following intervals is the function $f$ given by $f(x) = x^{100} + \sin x - 1$ decreasing?
    \begin{itemize}
        \item[(A)] $(0, 1)$ \hfill (B) $\left(\frac{\pi}{2}, \pi\right)$ \hfill (C) $\left(0, \frac{\pi}{2}\right)$ \hfill (D) None of these
    \end{itemize}

    \item For what values of $a$ the function $f$ given by $f(x) = x^2 + ax + 1$ is increasing on $[1, 2]$?

    \item Let $I$ be any interval disjoint from $[-1, 1]$. Prove that the function $f$ given by $f(x) = x + \frac{1}{x}$ is increasing on $I$.

    \item Prove that the function $f$ given by $f(x) = \log \sin x$ is increasing on $\left(0, \frac{\pi}{2}\right)$ and decreasing on $\left(\frac{\pi}{2}, \pi\right)$.

    \item Prove that the function $f$ given by $f(x) = \log |\cos x|$ is decreasing on $\left(0, \frac{\pi}{2}\right)$ and increasing on $\left(\frac{3\pi}{2}, 2\pi\right)$.

    \item Prove that the function given by $f(x) = x^3 - 3x^2 + 3x - 100$ is increasing in $\mathbb{R}$.

    \item The interval in which $y = x^2 e^{-x}$ is increasing is:
    \begin{itemize}
        \item[(A)] $(-\infty, \infty)$ \hfill (B) $(-2, 0)$ \hfill (C) $(2, \infty)$ \hfill (D) $(0, 2)$
    \end{itemize}
\end{enumerate}

\newpage

\section*{6.6 Maxima and Minima}

In this section, we will use the concept of derivatives to calculate the maximum
or minimum values of various functions. In fact, we will find the \textit{turning points}
of the graph of a function and thus find points at which the graph reaches its
highest (or lowest) value \textit{locally}. The knowledge of such points is very useful
in sketching the graph of a given function. Further, we will also find the absolute
maximum and absolute minimum of a function that are necessary for the solution
of many applied problems.

Let us consider the following problems that arise in day-to-day life.

\begin{enumerate}
    \item[(i)] The profit from a grove of orange trees is given by
    \[
        P(x) = ax + bx^2
    \]
    where $a, b$ are constants and $x$ is the number of orange trees per acre.
    How many trees per acre will maximise the profit?

    \item[(ii)] A ball, thrown into the air from a building $60$ metres high,
    travels along a path given by
    \[
        h(x) = 60 + x - \frac{x^2}{60}
    \]
    where $x$ is the horizontal distance from the building and $h(x)$ is the height
    of the ball. What is the maximum height the ball will reach?

    \item[(iii)] An Apache helicopter of the enemy is flying along the path
    given by the curve
    \[
        f(x) = x^2 + 7
    \]
    A soldier, placed at the point $(1, 2)$, wants to shoot the helicopter when it
    is nearest to him. What is the nearest distance?
\end{enumerate}

In each of the above problems, there is something common, i.e., we wish to find
out the maximum or minimum values of the given functions. In order to tackle such
problems, we first formally define maximum or minimum values of a function,
points of local maxima and minima and tests for determining such points.

\subsection*{Definition 3}

Let $f$ be a function defined on an interval $I$. Then,

\begin{enumerate}
    \item[(a)] $f$ is said to have a \textit{maximum value} in $I$, if there exists
    a point $c$ in $I$ such that
    \[
        f(c) > f(x), \quad \text{for all } x \in I.
    \]
    The number $f(c)$ is called the maximum value of $f$ in $I$ and the point $c$
    is called a \textit{point of maximum value} of $f$ in $I$.

    \item[(b)] $f$ is said to have a \textit{minimum value} in $I$, if there exists
    a point $c$ in $I$ such that
    \[
        f(c) < f(x), \quad \text{for all } x \in I.
    \]
    The number $f(c)$, in this case, is called the minimum value of $f$ in $I$ and
    the point $c$ is called a \textit{point of minimum value} of $f$ in $I$.

    \item[(c)] $f$ is said to have an \textit{extreme value} in $I$, if there exists
    a point $c$ in $I$ such that $f(c)$ is either a maximum value or a minimum value
    of $f$ in $I$.

    The number $f(c)$, in this case, is called an \textit{extreme value} of $f$ in $I$
    and the point $c$ is called an \textit{extreme point}.
\end{enumerate}

\subsection*{Remark}

In Fig. 6.9 (a), (b) and (c), we have exhibited that graphs of certain particular
functions help us to find maximum value and minimum value at a point. In fact,
through graphs, we can even find maximum or minimum value of a function at a
point at which it is not even differentiable (Example 27).

\noindent\textbf{Visual Representation of Maxima and Minima}\\
\small\textit{(Insert Fig.\ 6.9 here if you have the image)}

\begin{figure}[H]
    \centering
    \includegraphics[width=\textwidth]{fig1.png}
    \caption{Graphs exhibiting maximum and minimum values.}
    \label{fig:task2_fig1}
\end{figure}

\newpage

\fancyhead[C]{\textit{Mathematics: Summary}}

\begin{center}
    {\color{cometblue}\hrule height 1.5pt}
    \vspace{0.3cm}
    \LARGE \textbf{Summary}
    \vspace{0.3cm}
    {\color{cometblue}\hrule height 1pt}
\end{center}

\vspace{0.5cm}

\section*{Summary}

\begin{itemize}[leftmargin=*]
    \item If a quantity $y$ varies with another quantity $x$, satisfying some rule $y=f(x)$, then
    \[
        \frac{dy}{dx}\; (\text{or } f'(x))
    \]
    represents the rate of change of $y$ with respect to $x$ and
    \[
        \left.\frac{dy}{dx}\right|_{x=x_0}\; (\text{or } f'(x_0))
    \]
    represents the rate of change of $y$ with respect to $x$ at $x=x_0$.

    \item If two variables $x$ and $y$ are varying with respect to another variable $t$, i.e.,
    $x=f(t)$ and $y=g(t)$, then by Chain Rule:
    \[
        \frac{dy}{dx}=\frac{\dfrac{dy}{dt}}{\dfrac{dx}{dt}}, \quad \text{if } \frac{dx}{dt}\neq 0.
    \]

    \item A function $f$ is said to be
    \begin{itemize}
        \item[(a)] \textbf{increasing} on an interval $(a,b)$ if
        \[
            x_1 < x_2 \text{ in } (a,b) \Rightarrow f(x_1) < f(x_2) \text{ for all } x_1,x_2 \in (a,b).
        \]
        Alternatively, if $f'(x)\ge 0$ for each $x$ in $(a,b)$.

        \item[(b)] \textbf{decreasing} on $(a,b)$ if
        \[
            x_1 < x_2 \text{ in } (a,b) \Rightarrow f(x_1) > f(x_2) \text{ for all } x_1,x_2 \in (a,b).
        \]
        Alternatively, if $f'(x)\le 0$ for each $x$ in $(a,b)$.

        \item[(c)] \textbf{constant} on $(a,b)$ if $f(x)=c$ for all $x \in (a,b)$, where $c$ is a constant.
        Alternatively, if $f'(x)=0$ for all $x$ in $(a,b)$.
    \end{itemize}

    \item The equation of the tangent at $(x_0,y_0)$ to the curve $y=f(x)$ is given by
    \[
        y-y_0=\left.\frac{dy}{dx}\right|_{x=x_0}(x-x_0).
    \]

    \item If $\dfrac{dy}{dx}$ does not exist at the point $(x_0,y_0)$, then the tangent at this point is
    parallel to the $y$-axis and its equation is $x=x_0$.

    \item If tangent to a curve $y=f(x)$ at $x=x_0$ is parallel to $x$-axis, then
    \[
        \left.\frac{dy}{dx}\right|_{x=x_0}=0.
    \]

    \item Equation of the normal to the curve $y=f(x)$ at a point $(x_0,y_0)$ is given by
    \[
        y-y_0=\frac{-1}{\left.\dfrac{dy}{dx}\right|_{x=x_0}}(x-x_0).
    \]

    \item If $\dfrac{dy}{dx}$ at the point $(x_0,y_0)$ is zero, then equation of the normal is $x=x_0$.

    \item If $\dfrac{dy}{dx}$ at the point $(x_0,y_0)$ does not exist, then the normal is parallel to $x$-axis
    and its equation is $y=y_0$.

    \item Let $y=f(x)$, $\Delta x$ be a small increment in $x$ and $\Delta y$ be the increment in $y$
    corresponding to the increment in $x$, i.e., $\Delta y=f(x+\Delta x)-f(x)$. Then $dy$ is given by
    \[
        dy=f'(x)\,dx \quad \text{or} \quad dy=\left(\frac{dy}{dx}\right)\Delta x.
    \]
    $dy$ is a good approximation of $\Delta y$ when $dx=\Delta x$ is relatively small and we denote it by
    $dy \approx \Delta y$.

    \item A point $c$ in the domain of a function $f$ at which either $f'(c)=0$ or $f$ is not differentiable
    is called a \textbf{critical point} of $f$.
\end{itemize}

\vspace{0.3cm}

\noindent\textbf{First Derivative Test:} Let $f$ be a function defined on an open interval $I$. Let $f$ be
continuous at a critical point $c$ in $I$. Then
\begin{enumerate}[leftmargin=*, label=(\roman*)]
    \item If $f'(x)$ changes sign from positive to negative as $x$ increases through $c$, i.e., if $f'(x)>0$
    at every point sufficiently close to and to the left of $c$, and $f'(x)<0$ at every point sufficiently close
    to and to the right of $c$, then $c$ is a point of \textit{local maxima}.

    \item If $f'(x)$ changes sign from negative to positive as $x$ increases through $c$, i.e., if $f'(x)<0$
    at every point sufficiently close to and to the left of $c$, and $f'(x)>0$ at every point sufficiently close
    to and to the right of $c$, then $c$ is a point of \textit{local minima}.

    \item If $f'(x)$ does not change sign as $x$ increases through $c$, then $c$ is neither a point of local maxima
    nor a point of local minima. In fact, such a point is called \textit{point of inflexion}.
\end{enumerate}

\vspace{0.3cm}

\noindent\textbf{Second Derivative Test:} Let $f$ be a function defined on an open interval $I$ and $c\in I$.
Let $f$ be twice differentiable at $c$. Then
\begin{enumerate}[leftmargin=*, label=(\roman*)]
    \item $x=c$ is a point of local maxima if $f'(c)=0$ and $f''(c)<0$. The value $f(c)$ is local maximum value of $f$.

    \item $x=c$ is a point of local minima if $f'(c)=0$ and $f''(c)>0$. In this case, $f(c)$ is local minimum value of $f$.

    \item The test fails if $f'(c)=0$ and $f''(c)=0$. In this case, we go back to the first derivative test and find
    whether $c$ is a point of maxima, minima or a point of inflexion.
\end{enumerate}

\vspace{0.3cm}

\noindent\textbf{Working rule for finding absolute maxima and/or absolute minima:}
\begin{enumerate}[leftmargin=*]
    \item \textbf{Step 1:} Find all critical points of $f$ in the interval, i.e., find points $x$ where either $f'(x)=0$
    or $f$ is not differentiable.

    \item \textbf{Step 2:} Take the end points of the interval.

    \item \textbf{Step 3:} At all these points (listed in Step 1 and 2), calculate the values of $f$.

    \item \textbf{Step 4:} Identify the maximum and minimum values of $f$ out of the values calculated in Step 3.
    This maximum value will be the absolute maximum value of $f$ and the minimum value will be the absolute minimum value of $f$.
\end{enumerate}

\end{document}
